\documentclass[portrait,final,a0paper]{baposter}
%\documentclass[a4shrink,portrait,final]{baposter}
% Usa a4shrink for an a4 sized paper.

\tracingstats=2

\usepackage{calc}
\usepackage{graphicx}
\usepackage{amsmath}
\usepackage{amssymb}
\usepackage{relsize}
\usepackage{multirow}
\usepackage{bm}
\usepackage{graphicx}
\usepackage{multicol}
\usepackage[utf8]{inputenc}
\DeclareUnicodeCharacter{00A0}{ }
\usepackage{pgfbaselayers}
\pgfdeclarelayer{background}
\pgfdeclarelayer{foreground}
\pgfsetlayers{background,main,foreground}

\usepackage{times}
\usepackage{helvet}
%\usepackage{bookman}
\usepackage{palatino}

\newcommand{\captionfont}{\footnotesize}

\selectcolormodel{cmyk}

\graphicspath{{images/}}

%%%%%%%%%%%%%%%%%%%%%%%%%%%%%%%%%%%%%%%%%%%%%%%%%%%%%%%%%%%%%%%%%%%%%%%%%%%%%%%%
%%%% Some math symbols used in the text
%%%%%%%%%%%%%%%%%%%%%%%%%%%%%%%%%%%%%%%%%%%%%%%%%%%%%%%%%%%%%%%%%%%%%%%%%%%%%%%%
% Format 
\newcommand{\Matrix}[1]{\begin{bmatrix} #1 \end{bmatrix}}
\newcommand{\Vector}[1]{\Matrix{#1}}
\newcommand*{\SET}[1]  {\ensuremath{\mathcal{#1}}}
\newcommand*{\MAT}[1]  {\ensuremath{\mathbf{#1}}}
\newcommand*{\VEC}[1]  {\ensuremath{\bm{#1}}}
\newcommand*{\CONST}[1]{\ensuremath{\mathit{#1}}}
\newcommand*{\norm}[1]{\mathopen\| #1 \mathclose\|}% use instead of $\|x\|$
\newcommand*{\abs}[1]{\mathopen| #1 \mathclose|}% use instead of $\|x\|$
\newcommand*{\absLR}[1]{\left| #1 \right|}% use instead of $\|x\|$

\def\norm#1{\mathopen\| #1 \mathclose\|}% use instead of $\|x\|$
\newcommand{\normLR}[1]{\left\| #1 \right\|}% use instead of $\|x\|$

%%%%%%%%%%%%%%%%%%%%%%%%%%%%%%%%%%%%%%%%%%%%%%%%%%%%%%%%%%%%%%%%%%%%%%%%%%%%%%%%
% Multicol Settings
%%%%%%%%%%%%%%%%%%%%%%%%%%%%%%%%%%%%%%%%%%%%%%%%%%%%%%%%%%%%%%%%%%%%%%%%%%%%%%%%
\setlength{\columnsep}{0.7em}
\setlength{\columnseprule}{0mm}


%%%%%%%%%%%%%%%%%%%%%%%%%%%%%%%%%%%%%%%%%%%%%%%%%%%%%%%%%%%%%%%%%%%%%%%%%%%%%%%%
% Save space in lists. Use this after the opening of the list
%%%%%%%%%%%%%%%%%%%%%%%%%%%%%%%%%%%%%%%%%%%%%%%%%%%%%%%%%%%%%%%%%%%%%%%%%%%%%%%%
\newcommand{\compresslist}{%
\setlength{\itemsep}{1pt}%
\setlength{\parskip}{0pt}%
\setlength{\parsep}{0pt}%
}


%%%%%%%%%%%%%%%%%%%%%%%%%%%%%%%%%%%%%%%%%%%%%%%%%%%%%%%%%%%%%%%%%%%%%%%%%%%%%%
%%% Begin of Document
%%%%%%%%%%%%%%%%%%%%%%%%%%%%%%%%%%%%%%%%%%%%%%%%%%%%%%%%%%%%%%%%%%%%%%%%%%%%%%

\begin{document}

%%%%%%%%%%%%%%%%%%%%%%%%%%%%%%%%%%%%%%%%%%%%%%%%%%%%%%%%%%%%%%%%%%%%%%%%%%%%%%
%%% Here starts the poster
%%%---------------------------------------------------------------------------
%%% Format it to your taste with the options
%%%%%%%%%%%%%%%%%%%%%%%%%%%%%%%%%%%%%%%%%%%%%%%%%%%%%%%%%%%%%%%%%%%%%%%%%%%%%%
% Define some colors
\definecolor{silver}{cmyk}{0,0,0,0.3}
\definecolor{yellow}{cmyk}{0,0,0.9,0.0}
\definecolor{reddishyellow}{cmyk}{0,0.22,1.0,0.0}
\definecolor{black}{cmyk}{0,0,0.0,1.0}
\definecolor{darkYellow}{cmyk}{0,0,1.0,0.5}
\definecolor{darkSilver}{cmyk}{0,0,0,0.1}

\definecolor{lightyellow}{cmyk}{0,0,0.3,0.0}
\definecolor{lighteryellow}{cmyk}{0,0,0.1,0.0}
\definecolor{lighteryellow}{cmyk}{0,0,0.1,0.0}
\definecolor{lightestyellow}{cmyk}{0,0,0.05,0.0}

%%
\typeout{Poster Starts}
\background{
  \begin{tikzpicture}[remember picture,overlay]%
    \draw (current page.north west)+(-2em,2em) node[anchor=north west] {\includegraphics[height=1.1\textheight]{silhouettes_background}};
  \end{tikzpicture}%
}

\newlength{\leftimgwidth}
\begin{poster}%
  % Poster Options
  {
  % Show grid to help with alignment
  grid=false,
  % Column spacing
  colspacing=1em,
  % Color style
  bgColorOne=lighteryellow,
  bgColorTwo=lightestyellow,
  borderColor=reddishyellow,
  headerColorOne=yellow,
  headerColorTwo=reddishyellow,
  headerFontColor=black,
  boxColorOne=lightyellow,
  boxColorTwo=lighteryellow,
  % Format of textbox
  textborder=roundedleft,
%  textborder=rectangle,
  % Format of text header
  eyecatcher=true,
  headerborder=open,
  headerheight=0.08\textheight,
  headershape=roundedright,
  headershade=plain,
  headerfont=\Large\textsf, %Sans Serif
  boxshade=plain,
%  background=shade-tb,
  background=plain,
  linewidth=2pt
  }
  % Eye Catcher
  {\includegraphics[width=10em]{docknroll}} % No eye catcher for this poster. (eyecatcher=no above). If an eye catcher is present, the title is centered between eye-catcher and logo.
  % Title
  {\sf %Sans Serif
  %\bf% Serif
   Développer, Tester et Livrer des \mbox{microservices à l'aide de Containers}}
  % Authors
  {\sf %Sans Serif
  % Serif
  \underline{Nicolas Herbaut} (LaBRI) \mbox{David Bourasseau et Maxime Peterlin (ENSEIRB-MATMECA)}
  }
  % University logo
  {% The makebox allows the title to flow into the logo, this is a hack because of the L shaped logo.
    \makebox[8em][r]{%
      \begin{minipage}{16em}
        \hfill
		\includegraphics[height=3.0em]{labri}
        \includegraphics[height=3.0em]{enseirb-matmeca}
		
      \end{minipage}
    }
  }

  \tikzstyle{light shaded}=[top color=baposterBGtwo!30!white,bottom color=baposterBGone!30!white,shading=axis,shading angle=30]

  % Width of left inset image
     \setlength{\leftimgwidth}{0.78em+8.0em}

%%%%%%%%%%%%%%%%%%%%%%%%%%%%%%%%%%%%%%%%%%%%%%%%%%%%%%%%%%%%%%%%%%%%%%%%%%%%%%
%%% Now define the boxes that make up the poster
%%%---------------------------------------------------------------------------
%%% Each box has a name and can be placed absolutely or relatively.
%%% The only inconvenience is that you can only specify a relative position 
%%% towards an already declared box. So if you have a box attached to the 
%%% bottom, one to the top and a third one which should be in between, you 
%%% have to specify the top and bottom boxes before you specify the middle 
%%% box.
%%%%%%%%%%%%%%%%%%%%%%%%%%%%%%%%%%%%%%%%%%%%%%%%%%%%%%%%%%%%%%%%%%%%%%%%%%%%%%
    %
    % A coloured circle useful as a bullet with an adjustably strong filling
    \newcommand{\colouredcircle}[1]{%
      \tikz{\useasboundingbox (-0.2em,-0.32em) rectangle(0.2em,0.32em); \draw[draw=black,fill=baposterBGone!80!black!#1!white,line width=0.03em] (0,0) circle(0.18em);}}


  \headerbox{Contribution}{name=contribution,column=0,row=0}{
  \vspace{0.3em}
  
   Nous exposons un retour d'expérience sur le déroulé d'un projet étudiant, visant à intégrer des microservices développés par des équipes différentes.
   
   Nous documentons également comme nous avons utilisé Docker comme unité de déploiement pour les mises en production et comme unité de développement pour les tests et l'intégration continue.
   
   \vspace{1em}
   \textbf{Mots clés}: \textit{Docker, Architecture Microservice, Architecture RestFul, Intégration Continue, Jenkins, Tests Unitaires, Tests d'intégration, Java, Mock}


 }
   \headerbox{Le Project SnapMail}{name=project,column=1,span=2,row=0}{

   \begin{multicols}{2}
			\begin{minipage}{0.5\textwidth}
				\includegraphics[width=\textwidth]{snapmail}
				\begin{center}\textbf{Un serveur mail à la maison qui héberge vos pièces jointes}
				\vspace{2em}
				\end{center}
			\end{minipage}
			Le projet est conçu comme un greffon à un autre projet de réseaux social distribué sur les passerelles résidentielles.
			L'utilisateur configure son client mail avec le serveur SMTP de la passerelle, afin d'y faire transiter tout son courrier sortant. 
			
			La passerelle va \textit{détacher} les pièces jointes, les stocker sur la passerelles et les remplacer dans le mail par des liens web. Ces liens pointent vers les contenus originaux ainsi que vers des versions optimisées des videos (streaming adaptatif) et photo (converties et redimentionnées)
	\end{multicols}
    }
 
   \headerbox{Concepts des Microservices}{name=microdesc,column=0,row=0,below=contribution}{
  \vspace{0.3em}
  
  \center{
   \includegraphics[width=\textwidth]{msprinciples}
   }
	


 }
   \headerbox{Mise en place des microservices}{name=micro,column=1,span=2,row=0,below=project}{
   
   
 }
 
   \headerbox{Tests \& intégration continue}{name=cidesc,column=0,row=0,below=microdesc}{
  
  
   \center{
   \includegraphics[width=\textwidth]{tests}
   }
   
 }
   \headerbox{Intégration Continue}{name=ci,column=1,span=2,row=0,below=micro}{
  
  
   Lorem ipsum dolor sit amet, consectetur adipiscing elit. Phasellus maximus ac est non mollis. Sed nibh eros, vestibulum at libero nec, accumsan elementum risus. Curabitur non ex pulvinar, laoreet est vitae, efficitur quam. Sed ut eros pharetra, vestibulum est vel, elementum ex. Etiam finibus in velit eu viverra. Nunc tincidunt lectus nec dictum interdum. Nunc tincidunt porta hendrerit. Proin elementum id elit nec lobortis. Quisque id sodales felis, accumsan finibus turpis. Pellentesque vitae odio orci. Curabitur in blandit nisi.

In hac habitasse platea dictumst. Integer condimentum malesuada risus et convallis. Sed sit amet augue non est tristique eleifend et et enim. Donec sit amet hendrerit eros. In ac eleifend orci, ac condimentum purus. Nunc mi est, rutrum sit amet erat blandit, congue ultrices turpis. Vestibulum auctor ac metus quis eleifend. In id accumsan purus, et suscipit nisi. Etiam mollis malesuada mi eget tristique. Duis vitae diam sit amet eros faucibus eleifend. Morbi consectetur placerat eros, sit amet tincidunt mi pulvinar non. 
\vspace{0.3em}
 }
 
   \headerbox{Références}{name=reference,column=0,row=0,below=cidesc}{
    
    [1] \textit{Blog} Microservice, de Martin Fowler http://bit.ly/1dI7ZJQ
    
    [2] \textit{Livre} \underline{Building Microservices} Designing Fine-Grained Systems, Sam Newman, O'Reilly Media 215
    
    [3] \textit{Podcast} Software Engineering Radio, Episode 213: James Lewis on Microservices, http://bit.ly/1tkIbeN
   
    
  
  
  \vspace{0.3em}
  
 }
   \headerbox{Remerciements}{name=ack,column=1,span=2,row=0,below=ci}{
   merci
  \vspace{0.3em}
  
 }
 




\end{poster}

\end{document}
